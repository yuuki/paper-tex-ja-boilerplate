\documentclass[submit,techreq]{ipsj}
% (journal) \documentclass[submit]{ipsj}

\usepackage{amsmath,amssymb,amsfonts}
\usepackage[dvipdfmx]{graphicx}
\usepackage[hyphens]{url}

\def\Underline{\setbox0\hbox\bgroup\let\\\endUnderline}
\def\endUnderline{\vphantom{y}\egroup\smash{\underline{\box0}}\\}
\def\|{\verb|}
% (journal) \setcounter{page}{1}

\begin{document}

% 和文表題
\title{IPSJスタイルの日本語LaTeXテンプレート}

% 英文表題
\etitle{LaTeX Template in Japanese for IPSJ style}

% 所属ラベルの定義
\paffiliate{SAKURA}{さくらインターネット株式会社 さくらインターネット研究所
SAKURA internet Research Center, SAKURA internet Inc.,
Ofukatyo, Kitaku, Osaka 530-0011 Japan
}
\paffiliate{KYOTOU}{京都大学情報学研究科,
Graduate School of Infomatics, Kyoto University,
Kyoto 606--8501, Japan}

\author{坪内 佑樹}{Yuuki Tsubouchi}{SAKURA,KYOTOU}[y-tsubouchi@sakura.ad.jp,y-tsubouchi@net.ist.i.kyoto-u.ac.jp]

% 和文内容梗概
\begin{abstract}
\end{abstract}

\begin{jkeyword}
\end{jkeyword}

% 英文内容梗概
\begin{eabstract}
\end{eabstract}

\begin{ekeyword}
\end{ekeyword}

\maketitle

\section{はじめに}
\label{sec:introduction}

\bibliographystyle{style/ipsjsort}
\bibliography{references}

\end{document}
